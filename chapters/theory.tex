\documentclass[class=scrbook, crop=false]{standalone}
\usepackage[subpreambles=true]{standalone}
\ifstandalone
    \input{../settings+/settings}
\fi

% ----------------------------------------------------------------------------
%                               Theoretical Background
% ----------------------------------------------------------------------------
\begin{document}

\ifstandalone
    \selectlanguage{ngerman}  % Toggle ON/OFF

    % Language-specific settings that change automatically
    \input{settings+/language}
\fi

\chapter{Theoretical Background}
\label{Chapter::Theoretical_Background} % Outline text
This chapter provides background knowledge to understand the concepts in the thesis.

% Background topics that are necessary to understand your thesis
\section{Theoretical Topic}
\label{Section::Theoretical_Topic}
    Text..

\section{Examples for Your Document}
    
    \subsection{Example of Glossary}
    \gls{Biofouling}
    
    \subsection{Example of Symbols}
    Examples of the use of symbols in a document. Symbols and also acronyms are introduced in a table at the begin of the document. To jump to this table, simply click on the symbol or acronym, it provides a hyperlink to the table.\\
    \gls{symb:height}
    \gls{symb:energy}
    \gls{symb:A}
    \gls{symb:B}
    \gls{symb:C}
    \gls{symb:D}
    \gls{symb:E}
    \gls{symb:F}
    \gls{symb:G}
    \gls{symb:H}
    \gls{symb:I}
    \gls{symb:J}
    \gls{symb:K}
    \gls{symb:L}
    \gls{symb:M}
    \gls{symb:N}
    \gls{symb:O}
    \gls{symb:P}
    \gls{symb:Q}
    \gls{symb:R}
    \gls{symb:S}
    \gls{symb:T}
    \gls{symb:U}
    \gls{symb:V}
    \gls{symb:W}
    \gls{symb:X}
    \gls{symb:Y}
    \gls{symb:Z}
    
    \subsection{Example of Acronmys}
    First use of the Acronym \gls{aA} prints long version with short version in brackets, second and following use will print only the short version: \gls{aA}
    
    \subsection{Special Characters}
    Registered: \TReg\\
    Copyright: \TCop\\
    Trademark: \TTra\\
    
    Großzügig Gréànauôbl Ç.\\
    !"§/()=?öäüß-.,<>\\
    
    \subsection{Example of Lists}
    \begin{itemize}
      \item List entries start with the \verb|\item| command!
      \item Individual entries are indicated with a black dot, a so-called bullet.
      \item The text in the entries may be of any length.
    \end{itemize}
    
    \subsection{Blind Text}
    \Blindtext
    
    \subsection{Example of an Equation}
    \begin{equation}
        \label{eq::Central_Projection}
            \left[\begin{array}{@{}c@{}} {}^c x_{\overline{p}} \\ {}^c y_{\overline{p}} \\ 1 \end{array} \right] =
            \underbrace{\left[\begin{array}{cccc}
                c & 0 & 0 & 0 \\
                0 & c & 0 & 0 \\
                0 & 0 & 1 & 0
            \end{array} \right]}_{\text{projection matrix}}
            \left[\begin{array}{@{}c@{}} {}^k X_p \\ {}^k Y_p \\ {}^k Z_p \\ 1 \end{array} \right] =
            \left[\begin{array}{@{}c@{}} c \cdot {}^k X_p \\ c \cdot {}^k Y_p \\ {}^k Z_p \end{array} \right] =
            \left[\begin{array}{@{}c@{}} c \dfrac{{}^k X_p}{{}^k Z_p} \\ c \dfrac{{}^k Y_p}{{}^k Z_p} \\ 1 \end{array} \right]
    \end{equation}
    
    \subsection{Example of Figure}
    \begin{figure}[ht]
            \centering
            \includegraphics[width=\textwidth]{theory/Pinhole-Camera-Model-Multiple-View-Geometry-in-Computer-Vision.pdf}
            \caption[Mathematical model of a pinhole camera]{Mathematical model of a pinhole camera \cite{Multiple_View_Geometry_in_Computer_Vision}.}
            \label{fig::Pinhole_Camera_Model}
        \end{figure}
    

    \subsection{Example of Reference}
    Reference Figure \ref{fig::Pinhole_Camera_Model} and Equation \ref{eq::Central_Projection}.
    Reference source \cite{3D_introductorytechniques}. Also check out code~\ref{Code:Super Code}.

    \subsection{Example of a Table}
    \begin{table*}[ht]
        \centering
        \caption[Datasets for 3D reconstruction]{Datasets for stereoscopic 3D reconstruction with ground-truth information.}
        \label{Table::Dataset_attributes}
        \begin{tabular}{|c|c|c|c|c|}
            \hline
            ~                        & Dynamic & Truth Type       & Remarks\\
            \hline
            \color{blue} Tsukuba      & --      & Manual           & The first data with GT\\
            \color{orange} Middlebury & --      & Structured Light & Most famous 3D data with GT \\
            \color{red} Hamlyn        & $X$     & partly available & Robotic surgery\\
            \color{cyan} Kitti        & $X$     & Lidar            & Autonomous driving\\
            \color{green} EndoVis     & $X$     & Structured Light & Robotic surgery\\
            \hline
        \end{tabular}
        \begin{tablenotes}
            \small
            \item * This is an example footnote for the table.
        \end{tablenotes}
    \end{table*}

    \subsection{Example of Inserted Code}
    \label{Section::Insert_Code}
    You can use the package listings to import code directly from files. 
    \lstinputlisting[language=C++,
                     caption=Descriptive Caption Text,
                     label=Code:Super Code]
    {theory/cuda_example.cu}

    In addition to this, it is also possible to directly input your code in \LaTeX.
    \begin{lstlisting}[language=C++,
                       caption=Another Descriptive Caption Text,
                       label=Code:Super Code2]
     // Some example
     callFunction();
    \end{lstlisting}

    \subsection{Example of ToDo-Notes}
    \label{Section::ToDo}
    With \verb|\todo{Example ToDo note}| you can define your own ToDo notes.
    \todo{Example ToDo note}

    We prefer the inline ToDo notes, but you can also use floating ones by using the command \verb|\todo[noinline]{Example ToDo note}|.
    % \todo[noinline]{Example ToDo note}

    You can add your ToDos to the Table of Contents with \verb|\todototoc|, or you can print a separate list of ToDo notes with \verb|\listoftodos|.
    If you prefer, you can also just color your text.

    \subsection{Coloring Text}
    \label{Section::Colors}
    You can color text by using predefined commands such as \verb|\red{text}|.
    Colors we provide are as follows: \red{red}, \green{green}, \blue{blue}, and \orange{orange}. Also, feel free to make up your own colors and add them to the \verb|settings+/variables.tex| file. This can be done in a few easy steps:
    \begin{itemize}
        \item Name a color and define it (see \url{https://latexcolor.com/} for ideas)\\
              \verb|\definecolor{dollarbill}{rgb}{0.52, 0.73, 0.4}|
        \item Optional: Create a shorthand command with the new color\\
              \verb|\newcommand\myColor[1]{\textcolor{dollarbill}{\textbf{#1}}}|
        \item Done! Color a text with the new color\\
              \verb|\textcolor{dollarbill}{Text}|\\
              or via shorthand\\
              \verb|\myColor{Text}|
    \end{itemize}

% You can use this to add content for standalone documents if you like
% In this case we would like to show the references.
\ifstandalone
    % Bibliography
    \printbibliography[heading=bibintoc]                         \cleardoublepage

% ----------------------------------------------------------------------------
% Appendix and Glossary
% ----------------------------------------------------------------------------
%     \pagenumbering{Alph} % A, B, C..

% %     % Appendix
%     \input{chapters/appendix}                                          \clearpage

% %     % Symbol list also counts as a glossary object
%     \printglossary[type=main]  % main glossary

% %     % Either print all entries or only used entries for all lists
%     \glsaddallunused
\fi

\end{document}
